%
% Hello! Here's how this works:
%
% You edit the source code here on the left, and the preview on the
% right shows you the result within a few seconds.
%
% Bookmark this page and share the URL with your co-authors. They can
% edit at the same time!
%
% You can upload figures, bibliographies, custom classes and
% styles using the files menu.
%
% If you're new to LaTeX, the wikibook at
% http://en.wikibooks.org/wiki/LaTeX
% is a great place to start, and there are some examples in this
% document, too.
%
% We're still in beta. Please leave some feedback using the link at
% the top left of this page. Enjoy!
%
\documentclass[12pt]{article}

\usepackage[portuguese]{babel}
\usepackage[utf8]{inputenc}
\usepackage{amsmath}
\usepackage{graphicx}
\usepackage[usenames,dvipsnames,svgnames,table]{xcolor}
\usepackage{listings}

\title{Implementação de um Ray Tracer}
\author{Miguel Correlo \\
\small{pg22780}\\
\small{correlomm@gmail.com}}
\lstdefinestyle{customc}{
  belowcaptionskip=1\baselineskip,
  breaklines=true,
  frame=L,
  xleftmargin=\parindent,
  language=C,
  showstringspaces=false,
  basicstyle=\footnotesize\ttfamily,
  keywordstyle=\bfseries\color{blue!40!black},
  commentstyle=\itshape\color{green!40!black},
  identifierstyle=\color{black},
  stringstyle=\color{orange},
}

\lstdefinestyle{customasm}{
  belowcaptionskip=1\baselineskip,
  frame=L,
  xleftmargin=\parindent,
  language=[x86masm]Assembler,
  basicstyle=\footnotesize\ttfamily,
  commentstyle=\itshape\color{green!40!black},
}

\lstset{escapechar=@,style=customc}
\begin{document}
\maketitle

\begin{abstract}
O {\it ray tracing} é uma das várias técnicas que existem para fazer {\it render} de imagem por compuador. A idéia acente por trás do {\it ray tracing} é reproduzir imagens a partir da realidade física existente no mundo real. As imagens são reproduzidas a partir da luz que geralmente vem de uma fonte luminosa e é decomposta em raios. Os raios são reproduzidos por computador simulando o percurso de um determinado raio que percorre até chegar aos nossos olhos desde de um ponto de luz. Permitindo assim simular de um modo artificial o que o nossos olhos vêm.

É evidente que não é tão simples quanto parece reproduzir essa realidade. É preciso de alguma forma acompanhar esses raios, o que é uma tarefa díficil pois a natureza reproduz essa realidade de um modo infinito o que leva a exigir um poder computacional que não temos à nossa disposição para a simular.

Uma das ideias fundamentais do {\it ray tracing} é precisamente devido à limitação computacional só se preocupar com os raios que atingem os nossos olhos e depois com alguns dos seus ressaltos. A segunda ideia passa pela geração das imagens que são geralmente matrizes de pixeis com uma resolução limitada. As duas ideias anteriores são a base para a implementação deste {\it ray tracer}. Neste trabalho serão abordados tópicos como o lançamento dos primeiros raios, sombras, reflexões, refrações, {\it anti-aliasing, motion blur, area lights, depth of Field, path Tracing with importance sampling e finalmente russian roulette}


\end{abstract}

\section{Introdução}
\vspace{10 mm}
\hspace{8 mm} asdsada a dsdasd


\subsection{Referências}

[1] Parallel-Split Shadow Maps for Large-scale Virtual Environments, da autoria de Fan Zhang, Hanqiu Sun, Leilei Xu, Lee Kit Lun \newline
[2] Cascaded Shadow Maps, autoria de Rouslan Dimitrov, NVidea Corporation \newline
[3] Sombras CG LEI, slides, autoria de António Ramires Fernandes \newline
[4] Notes On Implementation Of Trapezoidal Shadow Maps, autoria de Eugene K. Ressler \newline
[5] http://http.developer.nvidia.com/ \newline

\end{document}













